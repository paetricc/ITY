%%%%%%%%%%%%%%%%%%%%%%%%%%%%%%%%%%%%%
% Autor: Tomas Bartu                %
% Email: xbartu11@stud.fit.vutbr.cz %
% Datum: 7.3.2023                   %
%%%%%%%%%%%%%%%%%%%%%%%%%%%%%%%%%%%%%

\documentclass[a4paper,twocolumn,11pt]{article}[06-03-2022]

\usepackage[text={18.2cm, 25.2cm}, left=1.4cm, top=2.3cm]{geometry}
\usepackage[czech]{babel}
\usepackage{times}
\usepackage[IL2]{fontenc}
\usepackage[utf8]{inputenc}
\usepackage{amsthm}
\usepackage{amssymb}
\usepackage{amsmath}

\theoremstyle{definition}
\newtheorem{theorem}{Definice}
\newtheorem{sentence}{Věta}

\begin{document}
\begin{titlepage}
    \begin{center}
        {\Huge \textsc{Vysoké učení technické v Brně} \\
        \vspace{0.5em}
        \huge\textsc{Fakulta informačních technologií}}
        
        \vspace{\stretch{0.382}}

        {\LARGE Typografie a publikování\,--\,2. projekt \\
        \vspace{0.4em}
        Sazba dokumentů a matematických výrazů}
        
        \vspace{\stretch{0.618}}
    \end{center}
    { \Large 2023 \hfill Tomáš Bártů (xbartu11) }    
\end{titlepage}


\section*{Úvod}
V této úloze si vyzkoušíme sazbu titulní strany, matematických vzorců, prostředí a dalších textových struktur obvyklých pro technicky zaměřené texty\,--\,například Definice nebo rovnice \eqref{eq3} na straně. 
Pro vytvoření těchto odkazů používáme kombinace příkazů \verb|\label|, \verb|\ref|, \verb|\eqref| a \verb|\pageref|.
Před odkazy patří nezlomitelná mezera.
Pro zvýrazňování textu jsou zde několikrát použity příkazy \verb|\verb| a \verb|\emph|. 

Na titulní straně je použito prostředí \texttt{titlepage} a sázení nadpisu podle optického středu s využitím \emph{přesného} zlatého řezu. 
Tento postup byl probírán na přednášce. 
Dále jsou na titulní straně použity čtyři různé velikosti písma a mezi dvojicemi řádků textu je použito odřádkování se zadanou relativní velikostí 0,5\,em a 0,4\,em\footnote{Nezapomeňte použít správný typ mezery mezi číslem a jednotkou.}.

\section{Matematický text}
V této sekci se podíváme na sázení matematických symbolů a výrazů v plynulém textu pomocí prostředí \texttt{math}.
Definice a věty sázíme pomocí příkazu \verb|\newtheorem| s využitím balíku \texttt{amsthm}.
Někdy je vhodné použít konstrukci \verb|${}$| nebo \verb|\mbox{}|, která říká, že (matematický) text nemá být zalomen.

\begin{theorem}
    Zásobníkový automat (ZA) \emph{je definován jako sedmice tvaru $A = (Q, \Sigma, \Gamma, \delta, q_0, Z_0, F)$, kde}: 
    \begin{itemize}
        \item $Q$ je konečná množina vnitřních (řídicích) stavů,
        \item $\Sigma$ je konečná vstupní abeceda,
        \item $\Gamma$ je konečná zásobníková abeceda,
        \item $\delta$ je přechodová funkce $Q \times (\Sigma \cup \{\epsilon\}) \times \Gamma \rightarrow 2^{Q \times \Gamma^\ast}$,
        \item $q_0 \in Q$ je počáteční stav, $Z_0 \in \Gamma$ je startovací symbol zásobníku a $F \subseteq Q$ je množina koncových stavů.
    \end{itemize}
\end{theorem}
Nechť $P = (Q, \Sigma, \Gamma, \delta, q_0, Z_0, F)$ je ZA. \emph{Konfigurací} nazveme trojici $(q,w,\alpha) \in \Sigma^\ast \times \Gamma^\ast$, kde $q$ je aktuální stav vnitřního řízení, $w$ je dosud nezpracovaná část vstupního řetězce a $\alpha = Z_{i_1}, Z_{i_2} \ldots Z_{i_k}$ je obsah zásobníku.

\subsection{Podsekce obsahující definici a větu}
\begin{theorem}
    Řetězec $w$ nad abecedou $\Sigma$ je přijat ZA $A$ \emph{jestliže $(q_0, w, Z_0) \overset{\ast}{\underset{A}{\vdash}} (q_F, \epsilon, \gamma)$ pro nějaké $\gamma \in \Gamma^\ast$ a $q_F \in F$.
    Množina $L(A) = \{w \bigm| w \textrm{ je přijat ZA } A \} \subseteq \Sigma^\ast$ je jazyk přijímaný ZA.}
\end{theorem}
\begin{sentence}
    \emph{Třída jazyků, které jsou přijímány ZA, odpovídá} bezkontextovým jazykům.
\end{sentence}

\section{Rovnice}
Složitější matematické formulace sázíme mimo plynulý text pomocí prostředí \texttt{displaymath}.
Lze umístit i několik výrazů na jeden řádek, ale pak je třeba tyto vhodně oddělit, například příkazem \verb|\quad|.
\begin{displaymath}
    1^{2^3} \neq \Delta^1_{\Delta^2_{\Delta^3}}
    \quad
    y^{11}_{22} - \sqrt[9]{x + \sqrt[7]{y}}
    \quad
    x > y_1 \leq y^2
\end{displaymath}
V rovnici \eqref{eq2} jsou využity tři typy závorek s různou \textit{explicitně} definovanou velikostí.
Také nepřehlédněte, že nasledující tři rovnice mají zarovnaná rovnítka, a použijte k tomuto účelu vhodné prostředí.
\begin{eqnarray}
    \label{eq1}
    - \cos^2 \beta & = & \frac{\frac{\frac{1}{x} + \frac{1}{3}}{y} + 1000}{\prod\limits^8_{j=2} q_j} \\
    \label{eq2}
    \left ( \left \{ b \star [3 \div 4] \circ a \right \}^\frac{2}{3} \right ) & = & \log_{10} x \\
    \label{eq3}
    \int^b_a f(x)\,\mathrm{d}x & = & \int^d_c f(y)\,\mathrm{d}x
\end{eqnarray}
V této větě vidíme, jak vypadá implicitní vysázení limity $\lim_{m\to\infty} f(m)$ v normálním odstavci textu. Podobně je to i s dalšími symboly jako $\bigcup_{N\in\mathcal{M}}N$ či $\sum^m_{i=1}x^2_i$.
S vynucením méně úsporné sazby příkazem \verb|\limits| budou vzorce vysázeny v podobě $\lim\limits_{m\to\infty} f(m)$ a $\sum\limits^m_{i=1} x^4_i$.

\section{Matice}
Pro sázení matic se velmi často používá prostředí \texttt{array} a závorky (\verb|\left|, \verb|\right|). 
Prostředí \texttt{array} lze úspěšně využít i jinde, například na pravé straně následující rovnice. 
Kombinační číslo na levé straně vysázejte pomocí příkazu \verb|\binom|.
$$ \binom{n}{k} = \left \{
\begin{array}{c l}
    0                   & \textrm{pro } k < 0 \\
    \frac{n!}{k!(n-k)!} & \textrm{pro } 0 \leq k \leq n \\
    0                   & \textrm{pro } k > 0 
\end{array}
\right. $$

\end{document}

