%%%%%%%%%%%%%%%%%%%%%%%%%%%%%%%%%%%%%
% Autor: Tomas Bartu                %
% Email: xbartu11@stud.fit.vutbr.cz %
% Datum: 19.3.2023                  %
%%%%%%%%%%%%%%%%%%%%%%%%%%%%%%%%%%%%%

\documentclass[a4paper,twocolumn,11pt]{article}[06-03-2022]

\usepackage[text={18.2cm, 25.2cm}, left=1.4cm, top=2.3cm]{geometry}
\usepackage[czech]{babel}
\usepackage{times}
\usepackage[IL2]{fontenc}
\usepackage[utf8]{inputenc}
\usepackage{amsthm}
\usepackage{amssymb}
\usepackage{amsmath}
\usepackage[hidelinks]{hyperref}

\theoremstyle{definition}
\newtheorem{theorem}{Definice}

\newtheorem{sentence}{Věta}

\begin{document}
\begin{titlepage}
    \begin{center}
        {\Huge \textsc{Vysoké učení technické v Brně} \\[0.5em]
        \huge\textsc{Fakulta informačních technologií}}
        
        \vspace{\stretch{0.382}}

        {\LARGE Typografie a publikování\,--\,2. projekt \\[0.4em]
        Sazba dokumentů a matematických výrazů}
        
        \vspace{\stretch{0.618}}
    \end{center}
    { \Large 2023 \hfill Tomáš Bártů (xbartu11) }    
\end{titlepage}


\section*{Úvod}
V této úloze si vyzkoušíme sazbu titulní strany, matematických vzorců, prostředí a dalších textových struktur obvyklých pro technicky zaměřené texty\,--\,například Definice~\ref{def1} nebo rovnice~\eqref{eq3} na straně~\pageref{eq3}. 
Pro vytvoření těchto odkazů používáme kombinace příkazů \verb|\label|, \verb|\ref|, \verb|\eqref| a \verb|\pageref|.
Před odkazy patří nezlomitelná mezera.
Pro zvýrazňování textu jsou zde několikrát použity příkazy \verb|\verb| a \verb|\emph|. 

Na titulní straně je použito prostředí \texttt{titlepage} a sázení nadpisu podle optického středu s využitím \emph{přesného} zlatého řezu. 
Tento postup byl probírán na přednášce. 
Dále jsou na titulní straně použity čtyři různé velikosti písma a mezi dvojicemi řádků textu je použito odřádkování se zadanou relativní velikostí 0,5\,em a 0,4\,em\footnote{Nezapomeňte použít správný typ mezery mezi číslem a jednotkou.}.

\section{Matematický text}
V této sekci se podíváme na sázení matematických symbolů a výrazů v plynulém textu pomocí prostředí \texttt{math}.
Definice a věty sázíme pomocí příkazu \verb|\newtheorem| s využitím balíku \texttt{amsthm}.
Někdy je vhodné použít konstrukci \verb|${}$| nebo \verb|\mbox{}|, která říká, že (matematický) text nemá být zalomen.

\begin{theorem}\label{def1}
    Zásobníkový automat \emph{(ZA) je definován jako sedmice tvaru $A = (Q, \Sigma, \Gamma, \delta, q_0, Z_0, F)$, kde:} 
    \begin{itemize}
        \item $Q$ \emph{je konečná množina} vnitřních (řídicích) stavů,
        \item $\Sigma$ \emph{je konečná} vstupní abeceda,
        \item $\Gamma$ \emph{je konečná} zásobníková abeceda,
        \item $\delta$ \emph{je} přechodová funkce $Q \times (\Sigma \cup \{\epsilon\}) \times \Gamma \rightarrow 2^{Q \times \Gamma^\ast}$,
        \item $q_0 \in Q$ \emph{je} počáteční stav, $Z_0 \in \Gamma$ \emph{je} startovací symbol zásobníku \emph{a $F \subseteq Q$ je množina} koncových stavů.
    \end{itemize}
\end{theorem}
Nechť $P = (Q, \Sigma, \Gamma, \delta, q_0, Z_0, F)$ je ZA. \emph{Konfigurací} nazveme trojici $(q,w,\alpha) \in Q \times \Sigma^\ast \times \Gamma^\ast$, kde $q$ je aktuální stav vnitřního řízení, $w$ je dosud nezpracovaná část vstupního řetězce a $\alpha = Z_{i_1} Z_{i_2} \ldots Z_{i_k}$ je obsah zásobníku.

\subsection{Podsekce obsahující definici a větu}
\begin{theorem}\label{def2}
    Řetězec $w$ nad abecedou $\Sigma$ je přijat ZA \emph{A jestliže $(q_0, w, Z_0) \overset{\ast}{\underset{A}{\vdash}} (q_F, \epsilon, \gamma)$ pro nějaké $\gamma \in \Gamma^\ast$ a $q_F \in F$.
    Množina $L(A) = \{w \bigm| w \textrm{ je přijat ZA } A \} \subseteq \Sigma^\ast$ je jazyk přijímaný ZA A.}
\end{theorem}
\begin{sentence}
    \emph{Třída jazyků, které jsou přijímány ZA, odpovídá} bezkontextovým jazykům.
\end{sentence}

\section{Rovnice}
Složitější matematické formulace sázíme mimo plynulý text pomocí prostředí \texttt{displaymath}.
Lze umístit i několik výrazů na jeden řádek, ale pak je třeba tyto vhodně oddělit, například příkazem \verb|\quad|.
\begin{displaymath}
    1^{2^3} \neq \Delta^1_{\Delta^2_{\Delta^3}}
    \quad
    y^{11}_{22} - \sqrt[9]{x + \sqrt[7]{y}}
    \quad
    x > y_1 \leq y^2
\end{displaymath}
V rovnici \eqref{eq2} jsou využity tři typy závorek s různou \textit{explicitně} definovanou velikostí.
Také nepřehlédněte, že nasledující tři rovnice mají zarovnaná rovnítka, a použijte k~tomuto účelu vhodné prostředí.
\begin{eqnarray}
    \label{eq1}
    - \cos^2 \beta & = & \frac{\frac{\frac{1}{x} + \frac{1}{3}}{y} + 1000}{\prod\limits^8_{j=2} q_j} \\
    \label{eq2}
    \bigg ( \Big \{ b \star \big [3 \div 4 \big ] \circ a \Big \}^\frac{2}{3} \bigg ) & = & \log_{10} x \\
    \label{eq3}
    \int^b_a f(x)\,\mathrm{d}x & = & \int^d_c f(y)\,\mathrm{d}y
\end{eqnarray}
V této větě vidíme, jak vypadá implicitní vysázení limity $\lim_{m\to\infty} f(m)$ v normálním odstavci textu. Podobně je to i s dalšími symboly jako $\bigcup_{N\in\mathcal{M}}N$ či $\sum^m_{i=1}x^2_i$.
S vynucením méně úsporné sazby příkazem \verb|\limits| budou vzorce vysázeny v podobě $\lim\limits_{m\to\infty} f(m)$ a $\sum\limits^m_{i=1} x^4_i$.

\section{Matice}
Pro sázení matic se velmi často používá prostředí \texttt{array} a závorky (\verb|\left|, \verb|\right|).
\begin{displaymath}
    \mathbf{B} = \left |
    \begin{array}{c c c c}
        b_{11} & b_{12} & \cdots & b_{1n} \\
        b_{21} & b_{22} & \cdots & b_{2n} \\
        \vdots & \vdots & \ddots & \vdots \\
        b_{m1} & b_{m2} & \cdots & b_{mn}
    \end{array}
    \right | = \left |
    \begin{array}{c c}
        t & u \\
        v & w \\
    \end{array}
    \right |
    = tw -uv
\end{displaymath}

\begin{displaymath}
    \mathbb{X} = \mathbf{Y} \Longleftrightarrow \left [
    \begin{array}{c c c}
                  & \Omega + \Delta & \hat{\psi} \\
        \vec{\pi} & \omega          &
    \end{array}
    \right ] \neq 42
\end{displaymath}

Prostředí \texttt{array} lze úspěšně využít i jinde, například na pravé straně následující rovnice. 
Kombinační číslo na levé straně vysázejte pomocí příkazu \verb|\binom|.
\begin{displaymath}
    \binom{n}{k} = \left \{
    \begin{array}{c l}
        0                   & \textrm{pro } k < 0 \\
        \frac{n!}{k!(n-k)!} & \textrm{pro } 0 \leq k \leq n \\
        0                   & \textrm{pro } k > 0 
    \end{array}
    \right.
\end{displaymath}

\end{document}

