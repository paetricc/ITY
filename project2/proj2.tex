%%%%%%%%%%%%%%%%%%%%%%%%%%%%%%%%%%%%%
% Autor: Tomas Bartu                %
% Email: xbartu11@stud.fit.vutbr.cz %
% Datum: 6.3.2023                   %
%%%%%%%%%%%%%%%%%%%%%%%%%%%%%%%%%%%%%

\documentclass[a4paper,twocolumn,11pt]{article}[06-03-2022]

\usepackage[text={18.2cm, 25.2cm}, left=1.4cm, top=2.3cm]{geometry}
\usepackage[czech]{babel}
\usepackage{times}
\usepackage[IL2]{fontenc}
\usepackage[utf8]{inputenc}

\begin{document}

\begin{titlepage}
    \begin{center}
        {
        \Huge \textsc{Vysoké učení technické v Brně}\\
        \vspace{0.5em}
        \huge \textsc{Fakulta informačních technologií}
        }
        
        \vspace{\stretch{0.382}}

        {
        \LARGE Typografie a publikování\,--\,2. projekt\\
        \vspace{0.4em}
        Sazba dokumentů a matematických výrazů
        }
        
        \vspace{\stretch{0.618}}
    \end{center}
    {\Large 2023 \hfill Tomáš Bártů (xbartu11)}    
\end{titlepage}


\section*{Úvod}
V této úloze si vyzkoušíme sazbu titulní strany, matematických vzorců, prostředí a dalších textových struktur obvyklých pro technicky zaměřené texty například Definice nebo rovnice na straně. Pro vytvoření těchto odkazů používáme kombinace příkazů \verb|\label|, \verb|\ref|, \verb|\eqref| a \verb|\pageref|. Před odkazy patří nezlomitelná mezera. Pro zvýrazňování textu jsou zde několikrát použity příkazy \verb|\verb| a \verb|\emph|. 

Na titulní straně je použito prostředí titlepage a sázení nadpisu podle optického středu s využitím přesného zlatého řezu. Tento postup byl probírán na přednášce. Dále jsou na titulní straně použity čtyři různé velikosti písma a mezi dvojicemi řádků textu je použito odřádkování se zadanou relativní velikostí 0,5\,em a 0,4\,em\footnote{Nezapomeňte použít správný typ mezery mezi číslem a jednotkou.}.

\section{Matematický text}
V této sekci se podíváme na sázení matematických symbolů a výrazů v plynulém textu pomocí prostředí math. Definice a věty sázíme pomocí příkazu \verb|\newtheorem| s využitím balíku amsthm. Někdy je vhodné použít konstrukci ${}$ nebo \mbox{}, která říká, že (matematický) text nemá být zalomen. Zásobníkový automat (ZA) je definován jako sedmice tvaru, kde: je konečná množina vnitřních (řídicích) stavů, je konečná vstupní abeceda, je konečná zásobníková abeceda, je přechodová funkce, je počáteční stav, je startovací symbol zásobníku a je množina koncových stavů. Nechť je ZA. Konfigurací nazveme trojici, kde je aktuální stav vnitřního řízení, je dosud nezpracovaná část vstupního řetězce a je obsah zásobníku.

\subsection{Podsekce obsahující definici a větu}
Řetězec nad abecedou je přijat ZA jestliže pro nějaké a. Množina A je jazyk přijímaný ZA. Třída jazyků, které jsou přijímány ZA, odpovídá bezkontextovým jazykům.

\section{Rovnice}
Složitější matematické formulace sázíme mimo plynulý text pomocí prostředí displaymath. Lze umístit i několik výrazů na jeden řádek, ale pak je třeba tyto vhodně oddělit, například příkazem \verb|\quad|. V rovnici jsou využity tři typy závorek s různou explicitně definovanou velikostí. Také nepřehlédněte, že nasledující tři rovnice mají zarovnaná rovnítka, a použijte k tomuto účelu vhodné prostředí. V této větě vidíme, jak vypadá implicitní vysázení limity v normálním odstavci textu. Podobně je to i s dalšími symboly jako či. S vynucením méně úsporné sazby příkazem \verb|\limits| budou vzorce vysázeny v podobě a.

\section{Matice}
Pro sázení matic se velmi často používá prostředí array a závorky (\verb|\left|, \verb|\right|). Prostředí array lze úspěšně využít i jinde, například na pravé straně následující rovnice. Kombinační číslo na levé straně vysázejte pomocí příkazu \verb|\binom|.

\end{document}

