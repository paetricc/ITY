%%%%%%%%%%%%%%%%%%%%%%%%%%%%%%%%%%%%%
% Autor: Tomas Bartu                %
% Email: xbartu11@stud.fit.vutbr.cz %
% Datum: 20.3.2023                  %
%%%%%%%%%%%%%%%%%%%%%%%%%%%%%%%%%%%%%

\documentclass[a4paper,11pt]{article}[20-03-2022]

\usepackage[text={17cm, 24cm}, left=2cm, top=3cm]{geometry}
\usepackage[ruled,vlined,linesnumbered]{algorithm2e}
\usepackage[czech]{babel}
\usepackage[utf8]{inputenc}
\usepackage{times}
\usepackage[hidelinks]{hyperref}
\usepackage{multirow}
\usepackage{xcolor}
\usepackage{array}

\newcommand{\rowstyle}[1]{\gdef\currentrowstyle{#1}#1\ignorespaces}

\begin{document}
\catcode`\-=12
\color{red}
\begin{titlepage}
    \begin{center}
        \Huge \textsc{Vysoké učení technické v Brně}

        \huge \textsc{Fakulta informačních technologií}

        \vspace{\stretch{0.382}}

        \LARGE Typografie a publikování\,--\,3. projekt

        \Huge Tabulky a obrázky

        \vspace{\stretch{0.618}}
    \end{center}
    {\Large \today \hfill Tomáš Bártů}    
\end{titlepage}

\section{Úvodní strana}
Název práce umístěte do zlatého řezu a nezapomeňte uvést \uv{dnešní}    (today) datum a vaše jméno a příjmení.

\section{Tabulky}
Pro sázení tabulek můžeme použít buď prostředí\texttt{ tabbing }nebo prostředí\texttt{ tabular }.

\subsection{Prostředí\texttt{ tabbing }}
Při použití\texttt{ tabbing }vypadá tabulka následovně:
\begin{tabbing}
    Vodní melouny \quad \= Cena \quad \= Množství \kill
    \textbf{Ovoce}      \> \textbf{Cena} \> \textbf{Množství}\\
    Jablka              \> 25,90         \> 3\,kg\\
    Hrušky              \> 27,40         \> 2,5\,kq\\
    Vodní melouny       \> 35,--         \> 1 kus\\
\end{tabbing}
Toto prostředí se dá také použít pro sázení algoritmů, ovšem vhodnější je použít prostředí\texttt{ algorithm }nebo\texttt{ algorithm2e }(viz sekce~\ref{sec:3}).

\subsection{Prostředí\texttt{ tabular }}
Další možností, jak vytvořit tabulku, je použít prostředí\texttt{ tabular}. 
Tabulky pak budou vypadat jako\footnotemark:
\bigskip
\begin{table}[h]
    \centering
    \begin{tabular}{| c | c | c |}
                                                     \hline
             & \multicolumn{2}{c |}{\textbf{Cena}} \\\cline{2-3}
        Měna & nákup  & prodej                     \\\hline
        EUR  & 22,705 & 25,242                     \\
        GBP  & 25,931 & 28,828                     \\
        USD  & 21,347 & 23,732                     \\\hline
    \end{tabular}
    \caption{Tabulka kurzů k dnešnímu dni}
\end{table}
\bigskip
\begin{table}[h]
    \centering
    \begin{tabular}{| >{\bfseries} c | c |}
        \hline
        $A$ & $\neg A$ \\\hline
        P & N \\\hline
        O & O \\\hline
        X & X \\\hline
        N & P \\\hline
    \end{tabular}
    \begin{tabular}{| c | >{\bfseries} c | c | c | c | c |}
        \hline
        \multicolumn{2}{| c |}{\multirow{2}{*}{$A \wedge B$}} & \multicolumn{4}{ c |}{$B$} \\\cline{3-6}
        \multicolumn{2}{| c |}{} & \textbf{P} & \textbf{O} & \textbf{X} & \textbf{N}       \\\hline
        \multirow{4}{*}{$A$} & P & P          & O          & X          & N                \\\cline{2-6}
                             & O & O          & O          & N          & N                \\\cline{2-6}
                             & X & X          & N          & X          & N                \\\cline{2-6}
                             & N & N          & N          & N          & N                \\\cline{2-6}
                                                                                             \hline
    \end{tabular}
    \begin{tabular}{| c | >{\bfseries} c | c | c | c | c |}
        \hline
        \multicolumn{2}{| c |}{\multirow{2}{*}{$A \vee B$}} & \multicolumn{4}{ c |}{$B$} \\\cline{3-6}
        \multicolumn{2}{| c |}{} & \textbf{P} & \textbf{O} & \textbf{X} & \textbf{N}     \\\hline
        \multirow{4}{*}{$A$} & P & P          & P          & P          & P              \\\cline{2-6}
                             & O & P          & O          & P          & O              \\\cline{2-6}
                             & X & P          & P          & X          & X              \\\cline{2-6}
                             & N & P          & O          & X          & N              \\\cline{2-6}
                                                                                           \hline
    \end{tabular}
    \begin{tabular}{| c | >{\bfseries} c | c | c | c | c |}
        \hline
        \multicolumn{2}{| c |}{\multirow{2}{*}{$A \rightarrow B$}} & \multicolumn{4}{ c |}{$B$} \\\cline{3-6}
        \multicolumn{2}{| c |}{} & \textbf{P} & \textbf{O} & \textbf{X} & \textbf{N}            \\\hline
        \multirow{4}{*}{$A$} & P & P          & O          & X          & N                     \\\cline{2-6}
                             & O & P          & O          & P          & O                     \\\cline{2-6}
                             & X & P          & P          & X          & X                     \\\cline{2-6}
                             & N & P          & P          & P          & P                     \\\cline{2-6}
                                                                                                  \hline
    \end{tabular}
    \caption{Protože Kleeneho trojhodnotová logika už je \uv{zastaralá}, uvádíme si zde příklad čtyřhodnotové logiky}
\end{table}

\footnotetext{Kdyby byl problem s\texttt{ cline,} zkuste se poddívat třeba sem: http://www.abclinuxu.cz/tex/poradna/show/325037.}

\newpage

\section{Algoritmy} \label{sec:3}
Pokud budeme chtít vysázet algoritmus, můžeme použít prostředí\texttt{ algorithm\footnote{Pro nápovědu, jak zacházet s prostředím\texttt{ algorithm,} můžeme zkusit tuhle stránku:\\http://ftp.cstug.cz/pub/tex/CTAN/macros/latex/contrib/algorithms/algorithms.pdf.} }nebo\texttt{ algorithm2e\footnote{Pro\texttt{ algorithm2e }zase tuhle: http://ftp.cstug.cz/pub/tex/CTAN/macros/latex/contrib/algorithm2e/doc/algorithm2e.pdf.}}. 
Příklad použití prostředí\texttt{ algorithm2e }viz Algoritmus~\ref{alg}.
\bigskip
\begin{algorithm}
    \DontPrintSemicolon
    \KwIn{$(X_{t-1}, u_t, z_t)$}
    \KwOut{$X_t$}
    $\overline{X_t} = X_t = 0$\;
    \For{$k = 1$ to $M$}{
        

    }
    \caption{\textsc{FastSLAM}\label{alg}}
\end{algorithm}

\section{Obrázky}
Do našich článků můžeme samozřejmě vkládat obrázky. Pokud je obrázkem fotografie, můžeme klidně použít bitmapový soubor. 
Pokud by to ale mělo být nějaké schéma nebo něco podobného, je dobrým zvykem takovýto obrázek vytvořit vektorově.

\end{document}