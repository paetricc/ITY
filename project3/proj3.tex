%%%%%%%%%%%%%%%%%%%%%%%%%%%%%%%%%%%%%
% Autor: Tomas Bartu                %
% Email: xbartu11@stud.fit.vutbr.cz %
% Datum: 20.3.2023                  %
%%%%%%%%%%%%%%%%%%%%%%%%%%%%%%%%%%%%%

\documentclass[a4paper,11pt]{article}[20-03-2022]

\usepackage[text={17cm, 24cm}, left=2cm, top=3cm]{geometry}
\usepackage{algorithm2e}
\usepackage[czech]{babel}
\usepackage[utf8]{inputenc}
\usepackage{times}
\usepackage[hidelinks]{hyperref}
\usepackage{xcolor}

\begin{document}
\color{green}
\begin{titlepage}
    \begin{center}
        \Huge \textsc{Vysoké učení technické v Brně}

        \huge \textsc{Fakulta informačních technologií}

        \vspace{\stretch{0.382}}

        \LARGE Typografie a publikování\,--\,3. projekt

        \Huge Tabulky a obrázky

        \vspace{\stretch{0.618}}
    \end{center}
    {\Large \today \hfill Tomáš Bártů}    
\end{titlepage}

\section{Úvodní strana}
Název práce umístěte do zlatého řezu a nezapomeňte uvést \uv{dnešní}    (today) datum a vaše jméno a příjmení.

\section{Tabulky}
Pro sázení tabulek můžeme použít buď prostředí\texttt{ tabbing }nebo prostředí\texttt{ tabular }.

\subsection{Prostředí\texttt{ tabbing }}
Při použití\texttt{ tabbing }vypadá tabulka následovně:
\begin{tabbing}
    
\end{tabbing}
Toto prostředí se dá také použít pro sázení algoritmů, ovšem vhodnější je použít prostředí\texttt{ algorithm }nebo\texttt{ algorithm2e }(viz sekce 3).

\end{document}